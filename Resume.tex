
\documentclass[11pt,a4paper,sans]{moderncv} 

\moderncvstyle{casual}    
\moderncvcolor{green} 

\usepackage[utf8]{inputenc}
\usepackage[russian]{babel}
\usepackage{hyperref}
\hypersetup{
    colorlinks=true,
    linkcolor=blue,
    filecolor=magenta,      
    urlcolor=blue
}


\setlength{\hintscolumnwidth}{3cm}

% adjust the page margins
\usepackage[scale=0.8]{geometry}

% personal data
\name{}{Глеб Пушкарев}
\title{Резюме \\ +7(952)666-13-10 \\ gleb.pushkarev@gmail.com \\
Github: \href{https://github.com/NelosG}{NelosG}}                           
\phone[mobile]{+7(952)666-13-10}                  
\email{gleb.pushkarev@gmail.com}
\social[github]{NelosG}                            

\makeatletter\renewcommand*{\bibliographyitemlabel}{\@biblabel{\arabic{enumiv}}}\makeatother

\begin{document}
\makecvtitle

\section{Профессия, специальность}
\cvitem{}{C++ разработчик}
\cvitem{}{Java разработчик}

\section{Неоконченное высшее образование}
\cvitem{2019-2023}{Университет ИТМО,}
\cvitem{}{Кафедра Компьютерных Технологий,} 
\cvitem{}{Специальность Прикладная математика и информатика}

\section{Опыт Работы}
\cvitem{2020}{Стажировка в ООО\grqqМРГТ\grqq(Системный администратор)}
    

\section{Знание Языков}
\cvitem{Русский}{Родной}
\cvitem{Английский}{B1}

\section{Ключевые навыки}
\cvitem{}{C++ 17, Java, Qt 5, CMake, Gradle, bash, Git, ООП, Паттерны проектирования}

\section{Учебные проекты}
\cvitem{C++}{\href{https://github.com/NelosG/intrusive_list_task}{intrusive\_list}, \href{https://github.com/NelosG/function_task}{function}, \href{https://github.com/NelosG/optional-task}{optional}, \href{https://github.com/NelosG/shared_ptr_task}{shared\_ptr}, \href{https://github.com/NelosG/signal_task}{signal},}

\cvitem{}{\href{https://github.com/NelosG/Bimap}{bimap}, \href{https://github.com/NelosG/Qt-Mandelbrot}{Mandelbrot} (multi threading + Qt),
\href{https://github.com/NelosG/Optimization-Methods/tree/main/Lab3}{Решение СЛАУ},}
\cvitem{}{различные алгоритмы и структуры данных для лабораторных работ,}
\cvitem{}{автоматы, графы и регулярные языки (курс дискретной математики)}
\cvitem{Java}{\href{https://github.com/NelosG/ITMO-KT/tree/master/Java\%20HM/FirstYear/2.1}{Парсер для алгебраических выражений}, \href{https://github.com/NelosG/ITMO-KT/tree/master/Java\%20HM/FirstYear/FastScanner}{Fast-Scanner},}
\cvitem{}{\href{https://github.com/NelosG/Optimization-Methods}{Реализация методов оптимизации}, \href{https://github.com/NelosG/ITMO-KT/tree/master/Java\%20HM/SecondYear/java-advanced/java-solutions/info/kgeorgiy/ja/pushkarev/hello}{Клиент Сервер},}
\cvitem{}{\href{https://github.com/NelosG/ITMO-KT/tree/master/Java\%20HM/SecondYear/java-advanced/java-solutions/info/kgeorgiy/ja/pushkarev/crawler}{Рекурсивный обход страниц с извлечением информации},}
\cvitem{}{\href{https://github.com/NelosG/ITMO-KT/tree/master/Java\%20HM/SecondYear/java-advanced/java-solutions/info/kgeorgiy/ja/pushkarev/implementor}{Implementor}(создает исходный файл с реализацией класса реализующим заданный интерфэйс или абстрактный класс),}
\cvitem{}{\href{https://github.com/NelosG/ITMO-KT/tree/master/Java\%20HM/SecondYear/java-advanced/java-solutions/info/kgeorgiy/ja/pushkarev/walk}{Обход файлов}, Markdown to HTML}
\cvitem{}{различные алгоритмы и структуры данных для лабораторных работ,}
\cvitem{}{автоматы, графы и регулярные языки (курс дискретной математики)}
\cvitem{Clojure}{\href{https://github.com/NelosG/ITMO-KT/tree/master/Clojure\%20HM}{Парсер для алгебраических выражений}, \href{https://github.com/NelosG/ITMO-KT/tree/master/Clojure\%20HM}{Операции с векторами и матрицами}}
\cvitem{JavaScript}{\href{https://github.com/NelosG/ITMO-KT/tree/master/Js\%20HM}{Парсер для алгебраических выражений}}
\cvitem{Prolog}{\href{https://github.com/NelosG/ITMO-KT/tree/master/Prolog\%20HM}{Поиск простых чисел}, \href{https://github.com/NelosG/ITMO-KT/tree/master/Prolog\%20HM}{Tree\_Map}}


 \section{Пройденные курсы}
\cvitem{}{Алгоритмы и структуры данных, Дискретная математика,}
\cvitem{}{Математический анализ, Линейная алгебра, Дифференциальные уравнения,}
\cvitem{}{Программирование на Java, Advanced Java, Курс C++,}
\cvitem{}{Теория вероятности, Архитектура ЭВМ, Операционные системы}
\cvitem{}{Методы оптимизации, Математическая логика}

 \section{Пожелания к предстоящей стажировке}
 \cvitem{Должность:}{Стажер-разработчик бэкенда(C++, Java)}

 \cvitem{Цели:}{Профессиональное саморазвитие}
 \cvitem{}{Возможность положительно влиять на результаты работы компании}
  \cvitem{}{Реализация интересных задач и проектов}
 
 
 \cvitem{Дополнительные пожелания:}{Присутствие инструктора или наставника}
\end{document}
